%Инклуды всякого

\documentclass[openany]{article}

\usepackage[T2A]{fontenc}
\usepackage[a4paper, left=20mm, right=20mm]{geometry}
\usepackage{caption}
\usepackage{pgfplots}
\usepackage{graphicx}
\usepackage{amsfonts}
\usepackage{amsmath}
\usepackage{graphicx}
\usepackage{fancyhdr}
\usepackage{amsmath}
\usepackage{tikz}
\usepackage{amssymb}
\newcommand{\Mod}[1]{\ (\mathrm{mod}\ #1)}
\usepackage{listings}
\usepackage{xcolor}
\date{Москва, 2022}
\title{Конспект к коллоквиуму по дискретной математике}
\author{Деревягин Александр \and{1 курс БПМИ 225-1}}

% Всякие мат штуки дополнительные

\newcommand{\F}{\mathbb{F}}
\renewcommand{\C}{\mathbb{C}}
\newcommand{\N} {\mathbb{N}}
\newcommand{\Z} {\mathbb{Z}}
\newcommand{\R} {\mathbb{R}}
\newcommand{\Q}{\mathbb{Q}}
\newcommand{\ord} {\mathop{\rm ord}}
\newcommand{\Ima}{\mathop{\rm Im}}
\newcommand{\Rea}{\mathop{\rm Re}}
\newcommand{\rk}{\mathop{\rm rk}}
\newcommand{\arccosh}{\mathop{\rm arccosh}}
\newcommand{\lker}{\mathop{\rm lker}}
\newcommand{\rker}{\mathop{\rm rker}}
\newcommand{\tr}{\mathop{\rm tr}}
\newcommand{\St}{\mathop{\rm St}}
\newcommand{\Mat}{\mathop{\rm Mat}}
\newcommand{\grad}{\mathop{\rm grad}}
\DeclareMathOperator{\spec}{spec}

% Всякое для ускорения
\renewcommand{\r}{\right}
\renewcommand{\l}{\left}
\newcommand{\Sum}[2]{\overset{#2}{\underset{#1}{\sum}}}
\newcommand{\Lim}[2]{\lim\limits_{#1 \rightarrow #2}}
\newcommand{\p}[2]{\frac{\partial #1}{\partial #2}}

% Заголовки
\newcommand{\task}[1] {\noindent \textbf{Задача #1.} \hfill}
\newcommand{\note}[1] {\noindent \textbf{Примечание #1.} \hfill}

% Пространтсва для задач
\newenvironment{proof}[1][Доказательство]{%
	\begin{trivlist}
		\item[\hskip \labelsep {\bfseries #1:}]
		\item \hspace{15pt}
}{
	$ \hfill\blacksquare $
	\end{trivlist}
	\hfill\break
}
\newenvironment{solution}[1][Решение]{%
	\begin{trivlist}
		\item[\hskip \labelsep {\bfseries #1:}]
		\item \hspace{15pt}
}{
	\end{trivlist}
}

\newenvironment{answer}[1][Ответ]{%
	\begin{trivlist}
		\item[\hskip \labelsep {\bfseries #1:}] \hskip \labelsep
}{
	\end{trivlist}
    \hfill
}

\begin{document}
   \maketitle
   \tableofcontents
   \newpage

   \section{Основные определения (множества, отношения, функции)}

   \subsection{Множества}

   \subsubsection{Определение}

   {\bfseriesМножество} -- это совокупность элементов.

   $X = \{a, b, c\}$

   (обычно обозначаются большими буквами)

   $a \in X$ -- элемент \textbf{лежит в множестве}, $b \notin X$ -- элемент \textbf{не лежит в множестве}.

   Множества можно задавать явно (через список элементов), а можно задавать с помощью условия ($Y = \{ y \in \N$| $y$ -- четно $\}$)

   $\emptyset$ -- \textbf{пустое множество}.

   Элементы множества -- любой объект.

   \subsubsection{Операции над множествами}

   Пусть $A, B$ -- множества. Тогда:

   $A \cup B = \{x |$ $x \in A$ или $x \in B$ $\}$

   $A \cap B = \{x |$ $x \in A$ и $x \in B$ $\}$

   $A \backslash B = \{x \in A |$ $x \notin B \}$

   \textbf{Дополнение} $\bar{A}$

   $A \triangle B =$ \textbf{симметрическая разность} = $\{x | (x \in A$ и $x \notin B)$ или $(x \notin A$ и $x \in B) \}$

   $A \subseteq X \Leftrightarrow \forall x (x \in A \rightarrow x \in X)$. $A$ -- \textbf{подмножество}, $X$ -- \textbf{надмножество}.

   $A = B \Leftrightarrow A \subseteq B$ и $B \subseteq A$.

   \subsubsection{Доказательство теоретико-множественных равенств}

   \textbf{Способ первый:} $A = B \Leftrightarrow A \subseteq B$ и $B \subseteq A$.

   \textbf{Утверждение 2.1} $(A \backslash B) \cup (B \backslash A) = (A \cup B) \backslash (A \cap B)$.

   \textbf{Доказательство:}

   $X = (A \backslash B) \cup (B \backslash A); Y =  (A \cup B) \backslash (A \cap B)$

   $X \subseteq Y:$ $x \in X$

   1. $x \in A \backslash B:$ $x \in A$ и $x \notin B$. Тогда $x \in A \cup B$, $x \notin A \cap B \rightarrow x \in Y$

   2. $x \in B \backslash A --$ аналогично

   $Y \subseteq X:$ $y \in Y, y \in A \cup B$

   $1. y \in A,$ $y \notin A \cup B \rightarrow y \notin B$. Тогда $y \in A \backslash B \rightarrow y \in X$

   $2. y \in B, y \notin A \cap B \rightarrow y \in B \backslash A$.

   \textbf{Способ второй:} расписать таблицу истинности

   \textbf{Утверждение 2.2} $(A \cup B) \backslash C = (A \backslash C) \cup (B \backslash C)$.

   \textbf{Доказательство:} 

   $X = (A \cup B) \backslash C; Y = (A \backslash C) \cup (B \backslash C)$

   \begin{tabular}{|c|c|c|c|c|c|c|c|}
       \hline
       $x \in A$ & $x \in B$ & $x \in C$ & $x \in A \backslash C$ & $x \in B \backslash C$ & $x \in A \cup B$ & $x \in X$ & $x \in Y$ \\
       \hline
       0 & 0 & 0 & 0 & 0 & 0 & 0 & 0 \\
       \hline
       0 & 0 & 1 & 0 & 0 & 0 & 0 & 0 \\
       \hline
       0 & 1 & 0 & 0 & 1 & 1 & 1 & 1 \\
       \hline
       0 & 1 & 1 & 0 & 0 & 1 & 0 & 0 \\
       \hline
       1 & 0 & 0 & 1 & 0 & 1 & 1 & 1 \\
       \hline
       1 & 0 & 1 & 0 & 0 & 1 & 0 & 0 \\
       \hline
       1 & 1 & 0 & 1 & 1 & 1 & 1 & 1 \\
       \hline
       1 & 1 & 1 & 0 & 0 & 1 & 0 & 0 \\
       \hline
   \end{tabular}

   \subsubsection{Дополнительные определения и парадокс Рассела}

   $A$ -- множество

   $2^{A}$ -- множество всех подмножеств

   \textbf{Пример:} $A = \{1, 2\}$ $2^{A} = \{\emptyset, \{1\}, \{2\}, \{1, 2\}\}$

   \textbf{Парадокс Рассела:} $U = \{x | x \notin x\}$.

   Вопрос: $U \in U$?

   Если да, то по определению $U$, $U \notin U$. Если нет, то т.к $U \notin U$, $U$ является элементом себя же. Противоречие.

   Потому множества по идее должны строиться за конечное число шагов и все дела, ну и наивная теория множеств разворачивается в огромный семестровый курс. Умные математики придумали аксиому Цермело -- Френкеля (ZF или ZFC) если вдруг будет интересно загуглить.

   \subsection{Отношения}

   \subsubsection{Декартово произведение и упорядоченные пары}

   $A, B$ -- множества

   Упорядоченная пара $(a, b)$ по Куратовскому -- это множество $\{a, \{a, b\}\}$

   \textbf{Упражнение:} Для определения упорядоченной пары по Куратовскому выполнено следующее свойство: $(a, b) = (a', b') \Leftrightarrow a = a'$ и $b = b'$.

   \textbf{Доказательство:}

   $\Leftarrow$ По определению. Пусть $a \neq a'$. Тогда пары не равны, т.к $\{a\} \notin (a', b') = \{\{a'\}, \{a', b'\}\}$. Пусть $a = a'$ но $b \neq b'$. Тогда пары не равны, т.к $\{a, b\} \notin (a', b') = \{\{a\}, \{a, b'\}\}$. Получается, что пары равны только тогда, когда $a' = a$ и $b' = b$.

   $\Rightarrow$ По определению. Т.к $(a, b) = (a', b')$, значит они состоят из одних и тех же элементов. Т.к в множествах $(a, b)$ и $(a', b')$ содержатся только по одному экземпляров одноэлементных множеств в каждом, $a = a'$. Также и для двухэлементных множеств: $\{a, b\} = \{a', b'\}, a = a' \Rightarrow b = b'$.

   $A \times B = \{(a, b) | a \in A, b \in B\}$ -- \textbf{декартово произведение}

   \subsubsection{Определение отношений}

   \textbf{Бинарное отношение} $R$ на множестве $A \times B$ -- это $R \subseteq A \times B$ такое, что если $x \in A$, $y \in B$ и $(x, y) \in R$, элементы соотносятся ($R(x, y) = 1, xRy$)

   Пример: $\R \times \R$, $x < y$ -- отношение.

   \textit{Лектор пытается визуализировать отношения...}

   \subsubsection{Композиция отношений}

   Пусть $R \subseteq A \times B, S \subseteq B \times C$. Тогда $(S \circ R) \subseteq A \times C$: $(a, c) \in S \circ R \Leftrightarrow \exists b \in B:$ $(a, b) \in R, (b, c) \in S$ ($aRb$ и $bSc$).

   \textbf{Теорема 2.3} Пусть $R \subseteq A \times B$, $S \subseteq B \times C$, $T \subseteq C \times D$. Тогда $T \circ (S \circ R) = (T \circ S) \circ R$. Иначе говоря, композиция отношений обладает свойством ассоциативности.

   \textbf{Доказательство:}

   $a \in A, d \in D$

   $(a, d) \in T \circ (S \circ R) \Leftrightarrow \exists z \in C: a(S \circ R)z$ и $zTd$ $\Leftrightarrow$ $\exists y \in B, z \in C: aRy$, $ySz$ и $zTd$.

   Правая часть расписывается аналогично:

   $(a, d) \in (T \circ S) \circ R \Leftrightarrow \exists y \in B: aRy$ и $y(T \circ S)d \Leftrightarrow \exists z \in C, y \in B: aRy$, $ySz$ и $zTd$.

   \subsection{Функции}

   \textbf{Функция f из A в B} -- это такое отношение $f \subseteq A \times B$, что $\forall a \in A$ в $f$ есть не более одной пары $(a, b)$, где $b \in B$.

   Обозначение: $(a. b) \in f$ или $afb$ $\Leftrightarrow f(a) = b$.

   Мы рассматриваем частичные функции, то есть они не полностью определены на $A$. Но

   \textbf{f на A и B тотальна}, если $Dom f = A$ (функция определена на всем множестве A). Тогда пишут $f: A \rightarrow B$.

   Запись $f: A \rightarrow B$ с подвохом: мы подразумеваем при подобной записи что $f$ тотальна, однако это может быть не так вне нашего курса, будьте бдительны.

   \textbf{Определение}: если $X \subseteq A$, то $f(X) = \{b \in B| \exists x \in X: f(x) = b\}$

   \textbf{Полный прообраз} $f^{-1}(Y) (Y \subseteq B) = \{a \in A| f(a) \in Y\}$.

   Пример: $f: \R \rightarrow \R$ $f(x) = x^2$

   $f^{-1}({0, 1}) = {-1, 0, 1}$

   $f({0, 1}) = {0, 1}$

   \subsubsection{Иньекции, сюрьекции, биекции}

   \textbf{Функция $f: A \rightarrow B$ называется инъекцией, если} $a_1 \neq a_2 \Rightarrow f(a_1) \neq f(a_2)$

   \textbf{Функция $f: A \rightarrow B$ называется сюръекцией, если} $\forall y \in B \exists x, f(x) = y$ (область значений функции есть все множество $B$).

   \textbf{Функция $f: A \rightarrow B$ называется биекцией, если} она одновременно и инъекция, и сюрьекция.

   \subsubsection{Композиция функций}

   Пусть $f: A \rightarrow B$ и $g: B \rightarrow C$. Тогда:

   $(g \circ f): A \rightarrow C$

   $(g \circ f)(a) = g(f(a))$

   Определение: $id_A: A \rightarrow A$, $id_A(a) = a$.

   Утверждение: $f: A \rightarrow B$. Тогда $f \circ id_A = id_B \circ f = f$.

   \subsubsection{Обратная функция}

   Пусть $f: A \rightarrow B$ -- биекция. Тогда $f^{-1}: B \rightarrow A$ определяется как $f^{-1}(b) = a \Leftrightarrow f(a) = b$.

   \textbf{Упражнение:} $f^{-1}$ -- биекция

   \textbf{Доказательство:}

   $1. f^{-1}$ -- иньекция. От противного: пусть $\exists a, b: a \neq b, f^{-1}(a) = f^{-1}(b)$. Тогда функция $f$ как будто принимает два различных значения при $f^{-1}(a)$, когда как по определению функции может принимать не более одного различного значения. Противоречие.

   $2. f^{-1}$ -- сюръекция. Вспомним, что $f: A \Rightarrow B$ тотальна, то есть $\forall a \in A \exists b: f(a) = b$. Из этого следует, что $\forall a \in A \exists b: f^{-1}(b) = a$.

   Функция $f^{-1}$ -- инъекция и сюръекция $\Rightarrow$ функция $f^{-1}$ -- биекция

   \textit{Примечание автора: обратная функция определяется только в том случае, когда f -- биекция. Если f не инъекция, то в каком-то значении b обратная функция будет принимать два различных значения, что противоречит определению функции. Если f не сюръекция, то обратная функция не будет тотальной, что противоречит уже самому определению обратной функции.}

   \textbf{Теорема 2.4} Пусть $f: A \rightarrow B$, $g: B \rightarrow A$. Пусть также $f \circ g = id_B$ и $g \circ f = id_A$. Тогда $f$ -- биекция и $g = f^{-1}$.

   \textbf{Доказательство:} 

   $1.$ f -- инъекция. От противного: пусть $f(a_1) = f(a_2), a_1 \neq a_2$. Тогда $a_1 = g(f(a_1)) = g(f(a_2)) = a_2$. Противоречие.

   $2.$ f -- сюръекция. $g(b) \in A$, $f(g(b)) = b$. То есть для каждого $b \in B$ мы нашли точку $a = g(b)$, причем она существует, т.к $g$ тотальна.

   Потому $f$ -- биекция. Пусть $f(a) = b$. Тогда $a = g(f(a)) = g(b) \Rightarrow g = f^{-1}$.

   \textbf{Следствие:} Если $f: A \rightarrow B$ и $g: B \rightarrow C$ -- биекция, то $g \circ f$ -- тоже биекция.

   Доказательство: Посмотрим на композицию функций $f^{-1} \circ g^{-1}$. 

   $g \circ f \circ f^{-1} \circ g^{-1} = id_C$ и $f^{-1} \circ g^{-1} \circ g \circ f = id_A$. Из теоремы следует, что $g \circ f$ -- биекция.

   \subsection{Отношения эквивалентности}

   Отношение $R$ на $A$ называют:

   \textbf{Рефлексивным}, если $\forall a \in A$, $aRa$.

   \textbf{Симметричным}, если $\forall a, b \in A$, ($aRb \Leftrightarrow bRa$).

   \textbf{Транзитивным}, если $\forall a, b, c$, ($aRb$ и $bRc$ $\Rightarrow$ $aRc$).

   Пример: отношение $a < b$ транзитивно, но не рефлексивно и не симметрично. Отношение $a + b = a * b$ симметрично, но не рефлексивно и не транзитивно.

   Отношение $R$ на $A$ называют \textbf{отношением эквивалентности}, если отношение $R$ рефлексивно, симметрично и транзитивно.

   Пример: Отношение $a = b$: рефлексивно ($a = a \forall a \in A$), симметрично ($a = b \Rightarrow b = a \forall a, b \in A$), транзитивно ($a = b. b = c \Rightarrow a = c \forall a, b, c \in A$).

   \textbf{Теорема 3.1} Если $R$ -- отношение эквивалентности на $A$, то $A = \vee_{i \in I}A_i$, $A_i \cap A_j = \emptyset (\forall i \neq j)$.

   \textbf{Доказательство:}

   $a \in A$

   $[a] = {b \in A | aRb}$ -- класс эквивалентности для $a$.

   1.) $a \in [a]$, т.к $aRa$.

   2.) Пусть $\neg (aRb) \Rightarrow [a] \cap [b] = \emptyset$.

   Действительно, если $x \in [a] \cap [b] = \emptyset$, то $aRb$ и  $bRc$
 $\Rightarrow$ $aRb$ -- противоречие.

   3.) Пусть $aRb$. Тогда $[a] = [b]$.

   Действительно, пусть $x \in [b]$, то есть $bRx$.

   Тогда $aRb$ и $bRx$ $\Rightarrow$ $aRx$ $\Rightarrow$ $x \in [a]$.

   Значит $[b] \subseteq [a]$

   Аналогично можно заключить, что $[a] \subseteq [b]$. Из этого следует, что классы совпадают.

   4.) $A$ есть объединение набора непересекающихся множеств (классов эквивалентности). Классы внутри пересекаться не могут, т.к если пересекаются, то по транзитивности это один и тот же класс.

   5.) Пусть $x, y \in [a]$. Тогда $aRx$ и $aRy$ $\Rightarrow$ $xRa$ и $aRy$ $\Rightarrow xRy$.

   6.) Если $x, y \in$ разным классам, то $\neg(xRy)$.

   От противного: пусть $x \in [a], y \in [b], [a] \neq [b], xRy$.

   Тогда $aRx$, $xRy$, $yRb$ $\Rightarrow$ $aRb$, то есть $[a] = [b]$.

   \section{Основы комбинаторики}

   \subsection{Основные определения}

   \subsubsection{Правила суммы и произведения}

   \textbf{Сумма:} $A = B \cup C$ и $B \cap C = \emptyset$, тогда $|A| = |B| + |C|$.

   \textbf{Утверждение:} $|A \cup B| = |A| + |B| - |A \cap B|$.

   \textbf{Доказательство:} $A \cup B = (A \backslash B) \cup B$, $(A \backslash B) \cap B = \emptyset \Rightarrow |A \cup B| = |A \backslash B| + |B|$. Т.к $A \backslash B = A \backslash (A \cap B)$, $|A \backslash B| = |A| - |A \cap B|$.

   $|A \cup B| = |A| + |B| - |A \cap B|$.

   \textbf{Произведение:} $|A \times B| = |A| * |B|$

   \textbf{Доказательство:} Индукция по $|B|$

   База: $|B| = 1$.

   $|A \times \{b\}| = |\{(a, b) | a \in A\}|$

   Переход: $|B| \rightarrow |B| + 1$

   $B' = B \cup \{b\}, b \notin B$.

   $A \times (B \cup \{b\}) = (A \times B) \cup (A \times \{b\})$

   \subsubsection{Основные комбинаторные величины}

   \textbf{Выборка k объектов из n объектов}

    \begin{tabular}{|c|c|c|}
        \hline
        Выборка: & Упорядоченная & Не упорядоченная \\
        \hline
        Без повторений & $\frac{n!}{(n-k)!}$ & $({}^n_k) = C_n^k = \frac{n!}{k!(n-k)!}$ \\
        \hline
        С повторениями & $n^k$ & $C_{n+k-1}^k$ \\
        \hline
    \end{tabular}

   \subsection{Бином Ньютона}

   \textbf{Теорема 4.1}: $(a + b)^n = \Sum{k=0}{n}({}^n_k)a^kb^{n-k}$

   \textbf{Доказательство:} Слагаемых $a^kb^{n-k} $в разложении $(a + b)^n$ будет столько же, сколько есть способов из $n$ скобок $(a + b)$ выбрать $a$ $k$ раз.

   \textbf{Следствие 4.2}:

   $1.) ({}_0^n) + ({}_1^n) + \ldots + ({}_n^n) = 2^n$

   $2.) ({}_0^n) - ({}_1^n) + ({}_2^n) - ({}_3^n) - \ldots - ({}_{2n-1}^{2n}) + ({}_{2n}^{2n}) = 0$ 

   \textbf{Доказательство}: 1. Разложение $(1 + 1)^n$, 2. Разложение $(1 - 1)^n$

   \textbf{Теорема 4.3} (полиномиальная теорема или обобщение бинома Ньютона на k слагаемых):

   $$(x_1 + x_2 + \ldots + x_k)^n = \sum\limits_{a_i \ge 0, \sum a_i = n} \frac{n!}{\alpha_1! * \alpha_2! * \ldots * \alpha_k!}x_1^{\alpha_1}x_2^{\alpha_2}\ldots x_k^{\alpha_k}$$

   \textbf{Доказательство}:

   \textbf{Обозначение:} $({}^{\hspace{0.6cm}n}_{\alpha_1, \alpha_2, \ldots \alpha_k}) = \frac{n!}{\alpha_1! \alpha_2! \ldots \alpha_k!}$

   Пусть мы хотим узнать коэффициент при $x_1^{\alpha_1}x_2^{\alpha_2}\ldots x_k^{\alpha_k}$. Это эквивалентно следующему числу:

   Пусть у нас есть алфавит $\{x_1, x_2, \ldots, x_n\}$, мы хотели бы найти в нем количество слов, в которых буква $x_i$ встречается ровно $\alpha_i$ раз.

   Составим такое слово сами. Пусть все $n$ позиций пусты. Среди них выберем $\alpha_1$ позицию где будет стоять $x_1$. Далее среди оставшихся позиций выберем позиции для $x_2$, среди оставшихся потом выберем позицию для $x_3$ и так далее.

   Количество способов составить уникальное слово в таком случае -- произведение $({}_{\alpha_1}^{n}) * ({}_{\alpha_2}^{n - \alpha_1}) * ({}_{\alpha_3}^{n - \alpha_1 - \alpha_2}) * \ldots$, ну то есть это можно вполне себе определить по индукции. Если раскрывать биномиальные коэффициенты по формуле, то $(n - k)!$ в знаменателях формул биномиальных коэффициентов будут сокращаться, что даст нам итоговую фомрулу.

   \section{Булевы функции}

   \subsection{Введение}

   \subsubsection{Основные понятия}

   \textbf{Определение:} Булева функция -- $f: \{0, 1\}^n \rightarrow \{0, 1\}$

   Пример:

   \begin{tabular}{|c|c|c|c|c|c|c|}
       \hline
       {} & {} & коньюнкция & дизьюнкция & xor & импликация & эквивалентность \\
       \hline
       $x$ & $y$ & $\wedge$ & $\vee$ & $\oplus$ & $\rightarrow$ & $\equiv$ \\
       \hline
       0 & 0 & 0 & 0 & 0 & 1 & 1 \\
       \hline
       0 & 1 & 0 & 1 & 1 & 1 & 0 \\
       \hline
       1 & 0 & 0 & 1 & 1 & 0 & 0 \\
       \hline
       1 & 1 & 1 & 1 & 0 & 0 & 1 \\
       \hline
   \end{tabular}

   Отрицание: $\neg x$, $\neg (0) = 1$, $\neg (1) = 0$

   \textbf{Утверждение 4.4}: Количество булевых функций от $n$ переменных равно $2^{2^n}$

   \textbf{Доказательство}: $n$ переменных могут принимать $2^n$ наборов значений в булевых функциях. Булева функция же условно выбирает себе наборы, на которых будет принимать единицу. Потому утверждение верно.

   \textbf{Упражнение:} Доказать следующие свойства булевых операций:

   \hspace{0.5cm}\parbox{11cm}{
       1. (коммутативность)

       \hspace{0.2cm}\hbox{
           $x \vee y = y \vee x$

           $x \wedge y = y \wedge x$

           $x \oplus y = y \oplus x$
       }

       2. (ассоциативность)

       \hspace{0.2cm}\parbox{11cm}{
           $x \vee (y \vee z) = (x \vee y) \vee z$

           $x \wedge (y \wedge z) = (x \wedge y) \wedge z$

           $x \oplus (y \oplus z) = (x \oplus y) \oplus z$
       }

       3. (дистрибутивность)

       \hspace{0.2cm}\parbox{11cm}{
           $(x \vee y) \wedge z = (x \wedge z) \vee (y \wedge z)$

           $(x \wedge y) \vee z = (x \vee z) \wedge (y \vee z)$

           $(x \oplus y) \wedge z = (x \wedge z) \oplus (y \wedge z)$
       }
   }

   \textbf{Доказательство:} расписать булеву табличку для каждого случая, либо просто рассмотреть случаи когда какая-либо переменная равна нулю/не равна нулю. Остается читателю в качестве упражнения.

   \subsubsection{Закон де Моргана}

   $\neg (x \wedge y) = \neg x \vee \neg y$

   $\neg (x \vee y) = \neg x \wedge \neg y$

   \subsubsection{Свойство импликации}

   $(x \rightarrow y) = (\neg x \rightarrow \neg y)$

   Потому мы можем доказывать утверждения от противного или использовать контрапозицию

   \subsubsection{Доказательство теоретико-множественных равенств}

   \textbf{Утверждение 4.5}: 

   $$(A_1 \cap A_2 \cap \ldots \cap A_n) \backslash (B_1 \cup B_2 \cup \ldots \cup B_n) = (A_1 \backslash B_1) \cap (A_2 \backslash B_2) \cap \ldots \cap (A_n \backslash B_n)$$

   \textbf{Доказательство}: Рассмотрим произвольный элемент $x$. ПУсть $a_i = x \in A_i$, $b_i = x \in B_i$. Тогда, $x \in A_i \backslash B_i$ будет соответствовать $a_i \wedge \neg b_i$, $x \in A_i \cap B_i$ будет соотоветствовать $a_i \wedge b_i$, $x \in A_i \cup B_i$ будет соответствовать $a_i \vee b_i$.

   Левая часть преобразуется в $a_1 \wedge a_2 \wedge \ldots \wedge a_n \wedge \neg b_1 \wedge \ldots \wedge \neg b_n$.

   Правая часть преобразуется в $(a_1 \wedge \neg b_1) \wedge (a_2 \wedge \neg b_2) \wedge \ldots \wedge (a_n \wedge \neg b_n)$

   Убираем скобки, переставляем элементы в правой части по комутативности коньюнкции, после чего получаем одно и то же.

   \subsubsection{Полные системы связок}

   Пусть $F$ это множество связок, $F = \{\neg, \wedge, \vee\}$

   \textbf{Определение:} Функция $f: \{0, 1\}^n \rightarrow \{0, 1\}$ выразима в системе связок $F$, если $\exists$ формула $\varphi$ под данной системой $F$:

   $$\forall (x_1, \ldots, x_n) \in \{0, 1\}^n: f(x_1, \ldots, x_n) = \varphi(x_1, \ldots, x_n)$$

   Формула $\varphi$ строится последовательно:

   \hspace{0.5cm}\parbox{11cm}{
       1. Переменная $x_i$ сама по себе является формулой

       2. Переменная $g(\varphi_1, \ldots, \varphi_n)$, где $g \in F$ и $\varphi_1, \varphi_2, \ldots, \varphi_n$ формулы -- тоже формула.

       3. Если $\varphi(x_1, x_2, \ldots, x_n)$ -- формула, то $\varphi(x_1, x_2, \ldots, x_n, x_{n + 1})$ тоже формула (где $x_{n + 1}$ фиктивная переменная, так мы умеем расширять количество аргументов у формулы).
   }

   Константы по умолчанию не являются формулами, их надо выражать из связок.

   \textbf{Определение} 1.) $[F]$ -- множество всех булевых функций, выразимых в $F$ (или замыкание $F$)

   2.) $F$ -- полная система связок, если $[F]$ -- все булевы функции ($P_2$).

   \textbf{Теорема 4.6}: $\{\neg, \wedge, \vee\}$ -- полная система связок

   \textbf{Доказательство}:

   Выразим функции $f$, равные единице только на одном конкретном наборе. Пусть такая функция $f(x_1, x_2, \ldots, x_n)$ равна единице на наборе $a_1, a_2, \ldots, a_n$. Тогда $f = y_1 \wedge y_2 \wedge \ldots \wedge y_n$, где $y_i = \neg x_i$, если $a_i = 0$ и $y_i = x_i$ иначе.

   Обозначим за $x^a = x$ если $a = 1$ и $\neg x$ если $a = 0$.

   То есть $f_{a_1, a_2, \ldots, a_n}(x_1, \ldots, x_n) = x_1^{a_1} \wedge x_2^{a_2} \wedge \ldots \wedge x_n^{a_n}$ -- функция, которая принимает 1 только на наборе $a_1, a_2, \ldots, a_n$.

   Пусть $f$ принимает 1 на некоторых наборах. Тогда $f = \bigvee\limits_{(a_1, \ldots, a_n) \in \{0, 1\}^n : f(a_1, a_2, \ldots, a_n) = 1} f_{a_1, a_2, \ldots, a_n} = \bigvee\limits_{(a_1, \ldots, a_n): f(a) = 1}x_1^{a_1}x_2^{a_2}\ldots x_n^{a_n}$

   Частный случай: тождественный ноль, мы можем его выразить как $x_1 \wedge \neg x_1$.

   Вообще, такое представление функции имеет название СДНФ или совершенная дизьюнктивная нормальная форма.

   \textbf{Дизьюнктивная Нормальная Форма (ДНФ)} -- представление функции $f(x_1, x_2, \ldots, x_n)$ как дизьюнкции коньюнктов.

   \textbf{Коньюнкт} -- $x_1^{a_1} \wedge x_2^{a_2} \wedge \ldots \wedge x_k^{a_k}$

   \subsection{Многочлен Жегалкина}

   \textbf{Теорема 4.9}: Система связок $\{1, \oplus, \wedge\}$ -- полная система связок

   \textbf{Доказательство:} Определим многочлен Жегалкина:

   $x_{i_1} \wedge x_{i_2} \wedge x_{i_k}$ -- моном.

   (0 и 1 -- тоже мономы)

   \textbf{Многочлен Жегалкина} -- многочлен вида $\bigoplus\limits_{(i_1,\ldots,i_k), k = 0 \ldots n} a_{i_1\ldots i_k}x_{i_1} \wedge x_{i_2} \wedge x_{i_k}$

   Пример: $1 \oplus (x \wedge y) \oplus (x \wedge y \wedge z)$

   Индукция по $n$:

   База: $n = 0$ -- 0 = $1 \oplus 1$

   Переход: $n \rightarrow n + 1$

   $f(x_1,x_2,\ldots,x_n,x_{n+1}) = (f(x_1,\ldots,x_n,0) \wedge (x_{n+1} \oplus 1)) \oplus (f(x_1, \ldots, x_n, 1) \wedge x_{n+1})$

   Подставляем вместо $x_{n+1}$ 0 и 1, получаем функци уже от $n$ переменных. По индукции, для них уже построены многочлены Жегалкина. Потому, подставим их вместо функций и приведем подобные, получим многочлен Жегалкина от $n + 1$ переменной.

   \subsection{Классы булевых функций}

   \subsubsection{Класс $L$}

   \textbf{Функция $f$ называется линейной, если} $f(x_1,\ldots, x_n) = a_0 \oplus a_1x_1 \oplus a_2x_2 \oplus \ldots \oplus a_nx_n$, где $a_i \in \{0, 1\}$

   $L = \{f \in P_2 |$ $f$ -- линейная$\}$

   Пример: $x_i \in L$, $x \oplus y \in L$, $0, 1 \in L$

   $x \wedge y \notin L, x \vee y \notin L$

   \textbf{Утверждение 5.1} $[L] = L$

   \textbf{Доказательство:} Индукция по построению формулы:

   Пусть $f_0(y_1, \ldots, y_k) f_1, f_2, \ldots, f_k \in L$

   Докажем, что $f_0 (f_1(x_1,\ldots,x_n),\ldots,f_k(x_1,\ldots,x_n)) = g \in L$

   Вспомним, что $g = a_0 \oplus a_1f_1 \oplus a_2f_2 \oplus \ldots \oplus a_kf_k$. Подставим $f_i$, раскроем скобки, приведем подобные и получим линейную функцию. Получается, что $g \in L$.

   \textbf{Утверждение 5.3} $L = [{\oplus, 1}]$

   \textbf{Доказательство:} по определению линейной функции.

   \textbf{Лемма 5.3} (о нелинейной функции): Пусть $f(x_1, \ldots, x_n) \notin L$. Тогда подставив вместо переменных функции $x_1, \ldots, x_n$ 0, $x$ и $y$ можно получить $g(x, y) \notin L$.

   \textbf{Доказательство:} $f(x_1,\ldots,x_n) = \ldots \oplus (x_{i_1} \wedge x_{i_2} \wedge \ldots \wedge x_{i_k}) \ldots$.

   Рассмотрим в многочлене Жегалкина мономы с количеством переменных $r \ge 2: x_{i_1} \wedge x_{i_2} \wedge \ldots \wedge x_{i_r}$. Подставим в $x_{i_1}$ $x$, а во все остальное $y$.

   $g(x, y) = x \wedge y \oplus ax \oplus by \oplus c \notin L$.

   \textbf{Следствие 5.4:} Пусть $f \notin L$. Тогда $x \wedge y \in [\{0, \neg x, f\}]$

   \textbf{Доказательство:} $g(x, y) = xy \oplus ax \oplus by \oplus c \in [\{0, f\}]$ (Лемма 5.4)

   Рассмотрим $g(x \oplus b, y \oplus a) = (x \oplus b) \wedge (y \oplus a) \oplus a(x \oplus b) \oplus b(y \oplus a) \oplus c = xy \oplus xa \oplus by \oplus ab \oplus ax \oplus ab \oplus by \oplus ab = xy \oplus ab \oplus c$.

   Если $ab \oplus c = 0$, то все хорошо и мы получили $xy$

   Иначе $\neg g(x \oplus b, y \oplus a) = xy$.

   \subsubsection{Класс S}

   Пусть $f(x_1, \ldots, x_n) \in P_2$. Тогда $f^*(x_1, \ldots, x_n) = \neg f(\neg x_1, \ldots, \neg x_n)$ -- \textbf{двойственная функция}.

   Пример: $(\neg x)^* = \neg(\neg(\neg x)) = \neg x$

   $(x \wedge y)^* = \neg((\neg x) \wedge (\neg y)) = x \vee y$.

   $(f^*)^* = f$

   \textbf{Лемма 5.5} (принцип двойственности):

   Пусть $f(x_1,\ldots,x_n) = f_0(f_1(x_1,\ldots,x_n),\ldots,f_k(x_1,\ldots,x_n))$. Тогда:

   $f^*(x_1,\ldots,x_n) = f_0^*(f_1^*(x_1,\ldots,x_n),\ldots,f_k^*(x_1,\ldots,x_n))$

   \textbf{Доказательство:}

   $$f^*(x_1,\ldots,x_n) = \neg f(f_1^*(\neg x_1, \ldots, \neg x_n), \ldots, f_k^*(\neg x_1, \ldots, \neg x_n)) = $$

   $$= \neg f_0(\neg f_1^*(x_1, \ldots, x_n), \ldots, \neg f_k^*(x_1, \ldots, x_n)) = f_0^*(f_1^*(x_1 \ldots x_n), \ldots, f_k^*(x_1 \ldots x_n))$$

   \textbf{Функция $f \in P_2$ называется самодвойственной}, если $f^* = f$.

   $S = \{f \in P_2 | f* = f\}$

   Пример: $x \in S, \neg x \in S, x \oplus y \oplus z \in S$

   $x \oplus y \notin S, x \wedge y, x \vee y \notin S$.

   \textbf{Следствие 5.6} $[S] = S$

   \textbf{Доказательство:} $x_i \in S$

   $f_0,\ldots,f_k \in S \Rightarrow f_0(f_1(x_1,\ldots,x_n),\ldots,f_k(x_1,\ldots,x_n)) = g(x_1,\ldots,x_n) \in S$

   $g^* = g$ по лемме 5.6

   \textbf{Лемма 5.7} (о несамодвойственной функции):

   Пусть $f(x_1, \ldots, x_n) \notin S$. Тогда подставляя вместо переменных функции $x, \neg x$, можно получить константу.

   \textbf{Доказательство:}

   ПУсть $f(x_1,\ldots,x_n) \neq \neg f(\neg x_1, \ldots, \neg x_n)$. Тогда есть какой-то набор $\alpha_1, \ldots, \alpha_n \in \{0, 1\}^n$ такой, что:

   $f(\alpha_1, \ldots, \alpha_n) = f(\neg \alpha_1, \ldots, \neg \alpha_n)$.

   Подставим вместо единиц в этом наборе $x$ и вместо нулей $\neg x$. Таким образом, получили новую функцию $g$ от одной переменной. Для неё будет справедливо следующее:

   $g(1) = f(\alpha_i) = f(\neg \alpha_i) = g(0)$.

   \subsubsection{Классы $T_0$ и $T_1$}

   \textbf{Класс $T_0 = \{f \in P_2 | f(0, \ldots, 0) = 0\}$}

   \textbf{Класс $T_1 = \{f \in P_2 | f(1, \ldots, 1) = 1\}$}

   ($f \in T_0$ -- функция, сохраняющая ноль; $f \in T_1$ -- функция, сохраняющая единицу).

   \textbf {Упражнение:} $[T_0] = T_0$, $[T_1] = T_1$

   \textbf {Доказательство:}

   Пусть $f_0, f_1, \ldots, f_k \in T_0$. Тогда $f_0 (f_1(x_1, \ldots, x_n), f_2(x_1, \ldots, x_n), \ldots, f_k(x_1, \ldots, x_n)) \in T_0$ так, как $f_0 (f_1 (0, \ldots, 0), f_2 (0, \ldots, 0), \ldots, f_k (0, \ldots, 0)) = f_0 (0, \ldots, 0) = 0$. 

   Аналогично доказывается и для $T_1$.

   \textbf {Лемма 5.8} 

   1. Если $f \notin T_0$, тогда $f (x, \ldots, x) = \{1, \neg x\}$. \textit{(т.к для $f (0, \ldots, 0)$ мы точно знаем что значение равно 1, а для $f (1, \ldots, 1)$ множество будет содержать в себе все возможные значения $f$).}

   2. Если $f \notin T_1$, тогда $f (x, \ldots, x) = \{0, \neg x\}$ \textit {(аналогично).}

   \subsubsection{Класс $M$}

   Для того, чтобы ввести класс монотонных функций нам нужно ввести понятие порядка. Скажем, что изначально $0 < 1$. Тогда:

   \textbf{Набор $(\alpha_1, \ldots, \alpha_n)$ меньше $(\beta_1, \ldots, \beta_n)$}, если $\forall i, \alpha_i \le \beta_i$.

   Пример: $(1, 0) \le (1, 1)$

   $(1, 0) \nleq (0, 1)$ (не сравнимы)

   $(0, 1) \nleq (1, 0)$ (не сравнимы)

   \textbf{$f \in P_2$ монотонная}, если $\forall \alpha_i, \beta_i$, $\alpha_i \le \beta_i \Rightarrow f(\alpha) \le f(\beta)$

   \textbf{Утверждение 5.9:} $[M] = M$

   \textbf{Доказательство:} $x_i \in M_i$

   $f_0, \ldots, f_k \in M$, $g(x_1, \ldots, x_n) = f_0(f_1(x_1, \ldots, x_n), \ldots, f_k(x_1, \ldots, x_n))$.

   Пусть $\alpha = (\alpha_1, \ldots, \alpha_n) \le (\beta_1, \ldots, \beta_n) = \beta$. Тогда $\forall 1 \le i \le k, f_i(\alpha) \le f_i(\beta) \Rightarrow f_0 ( f_i(\alpha) ) \le f_0 ( f_i (\beta) )$.

   \textbf{Лемма 5.10 (о немонотонной функции)}

   Пусть $f (x_1, \ldots, x_n) \notin M$. Тогда, подставляя вместо переменных 0, 1, x, можно получить $\neg x$.

   \textbf{Доказательство:} $\exists \alpha = (\alpha_1, \ldots, \alpha_n)$, $\exists \beta = (\beta_1, \ldots, \beta_n)$ такие, что $\alpha \le \beta$, но при этом $f(\alpha) = 1, f(\beta) = 0$ (т.к функция $\notin M$). 

   Построим новую функцию $g(x)$, полученная в результате подставления в $x_i$ значения $0, 1$ и $x$. Рассмотрим две группы индексов $i$: 

   \hspace{0.5cm}\parbox{12cm} {

   1. $\alpha_i = \beta_i$. Тогда поставим в $x_i$ значение $\alpha_i$.

   2. $\alpha_i = 0, \beta_i = 1$. Тогда поставим в $x_i$ переменную $x$.
   }

   При подстановке в $x$ значения 1 получим значение $g(1) = f(\beta) = 0$. При подстановке в $x$ значения 0 получим значение $g(0) = f(\alpha) = 1$. Получим то, что нам было нужно.

   \subsubsection{Критерий Поста}

   \textbf{Теорема 5.11:} $[F] = P_2 \Leftrightarrow F \nsubseteq L, F \nsubseteq T_0, F \nsubseteq T_1, F \nsubseteq S, F \nsubseteq M$

   \textbf{Доказательство:}

   $1. \Leftarrow$ От противного: пусть $F \subseteq C$, где $C$ это какой-то класс. Тогда $[F] \subseteq [C] = C$.

   $2. \Rightarrow$ Пусть $\exists f_L \notin F \backslash L$, $f_{T_0} \notin F \backslash T_0$, $f_{T_1} \notin F \backslash T_1$, $f_M \notin F \backslash M$, $f_S \notin F \backslash S$

   Из леммы 5.8 заметим, что мы можем выразить из функций не из $T_0$ и не из $T_1$ либо константы, либо отрицание $\neg x$. Пусть мы смогли выразить отрицание. Тогда по лемме о несамодвойственной функции мы также можем выразить 0 и 1. Пусть мы не смогли выразить отрицание. Тогда мы точно смогли выразить 0 и 1, потому по лемме о немонотонной функции, используя 0, 1 и $x$, мы можем выразить $\neg x$.

   По лемме о нелинейной функции, коньюнкция $x \wedge y \in [0, \neg x, f]$. Потому мы также можем выразить коньюнкцию, а из коньюнкции и отрицания можем выразить дизьюнкцию, потому мы получили полную систему связок.

   \subsection{Предполные классы}

   Пусть $F \subseteq P_2$ -- замкнутый класс ($[F] = F$)

   $F$ -- предполный в $P_2$, если $F \neq P_2$, но $\forall g \notin F$ $[F \cup {g}] = P_2$.

   \textbf{Теорема 6.1} (описание предполных классов) В $P_0$ есть ровно $5$ предполных классов: $S, L, M, T_0, T_1$.

   \textbf{Доказательство:}

   \begin{tabular}{|c|c|c|c|c|c|}
       {} & $T_0$ & $T_1$ & $M$ & $S$ & $L$ \\
       $T_0$ & $\times$ & 0 & $x \oplus y$ & $x \wedge y$ \\
       $T_1$ & 1 & $\times$ & $x \oplus y \oplus 1$ & 1 & $x \wedge y$ \\
       $M$ & 1 & 0 & $\times$ & 1 & $x \wedge y$ \\
       $S$ & $\neg x$ & $\neg x$ & $\neg x$ & $\times$ $MAJ(x_1, x_2, x_3)$ \\
   \end{tabular}
   \end{document}
